
% Default to the notebook output style

    


% Inherit from the specified cell style.




    
\documentclass[11pt]{article}

    
    
    \usepackage[T1]{fontenc}
    % Nicer default font (+ math font) than Computer Modern for most use cases
    \usepackage{mathpazo}

    % Basic figure setup, for now with no caption control since it's done
    % automatically by Pandoc (which extracts ![](path) syntax from Markdown).
    \usepackage{graphicx}
    % We will generate all images so they have a width \maxwidth. This means
    % that they will get their normal width if they fit onto the page, but
    % are scaled down if they would overflow the margins.
    \makeatletter
    \def\maxwidth{\ifdim\Gin@nat@width>\linewidth\linewidth
    \else\Gin@nat@width\fi}
    \makeatother
    \let\Oldincludegraphics\includegraphics
    % Set max figure width to be 80% of text width, for now hardcoded.
    \renewcommand{\includegraphics}[1]{\Oldincludegraphics[width=.8\maxwidth]{#1}}
    % Ensure that by default, figures have no caption (until we provide a
    % proper Figure object with a Caption API and a way to capture that
    % in the conversion process - todo).
    \usepackage{caption}
    \DeclareCaptionLabelFormat{nolabel}{}
    \captionsetup{labelformat=nolabel}

    \usepackage{adjustbox} % Used to constrain images to a maximum size 
    \usepackage{xcolor} % Allow colors to be defined
    \usepackage{enumerate} % Needed for markdown enumerations to work
    \usepackage{geometry} % Used to adjust the document margins
    \usepackage{amsmath} % Equations
    \usepackage{amssymb} % Equations
    \usepackage{textcomp} % defines textquotesingle
    % Hack from http://tex.stackexchange.com/a/47451/13684:
    \AtBeginDocument{%
        \def\PYZsq{\textquotesingle}% Upright quotes in Pygmentized code
    }
    \usepackage{upquote} % Upright quotes for verbatim code
    \usepackage{eurosym} % defines \euro
    \usepackage[mathletters]{ucs} % Extended unicode (utf-8) support
    \usepackage[utf8x]{inputenc} % Allow utf-8 characters in the tex document
    \usepackage{fancyvrb} % verbatim replacement that allows latex
    \usepackage{grffile} % extends the file name processing of package graphics 
                         % to support a larger range 
    % The hyperref package gives us a pdf with properly built
    % internal navigation ('pdf bookmarks' for the table of contents,
    % internal cross-reference links, web links for URLs, etc.)
    \usepackage{hyperref}
    \usepackage{longtable} % longtable support required by pandoc >1.10
    \usepackage{booktabs}  % table support for pandoc > 1.12.2
    \usepackage[inline]{enumitem} % IRkernel/repr support (it uses the enumerate* environment)
    \usepackage[normalem]{ulem} % ulem is needed to support strikethroughs (\sout)
                                % normalem makes italics be italics, not underlines
    

    
    
    % Colors for the hyperref package
    \definecolor{urlcolor}{rgb}{0,.145,.698}
    \definecolor{linkcolor}{rgb}{.71,0.21,0.01}
    \definecolor{citecolor}{rgb}{.12,.54,.11}

    % ANSI colors
    \definecolor{ansi-black}{HTML}{3E424D}
    \definecolor{ansi-black-intense}{HTML}{282C36}
    \definecolor{ansi-red}{HTML}{E75C58}
    \definecolor{ansi-red-intense}{HTML}{B22B31}
    \definecolor{ansi-green}{HTML}{00A250}
    \definecolor{ansi-green-intense}{HTML}{007427}
    \definecolor{ansi-yellow}{HTML}{DDB62B}
    \definecolor{ansi-yellow-intense}{HTML}{B27D12}
    \definecolor{ansi-blue}{HTML}{208FFB}
    \definecolor{ansi-blue-intense}{HTML}{0065CA}
    \definecolor{ansi-magenta}{HTML}{D160C4}
    \definecolor{ansi-magenta-intense}{HTML}{A03196}
    \definecolor{ansi-cyan}{HTML}{60C6C8}
    \definecolor{ansi-cyan-intense}{HTML}{258F8F}
    \definecolor{ansi-white}{HTML}{C5C1B4}
    \definecolor{ansi-white-intense}{HTML}{A1A6B2}

    % commands and environments needed by pandoc snippets
    % extracted from the output of `pandoc -s`
    \providecommand{\tightlist}{%
      \setlength{\itemsep}{0pt}\setlength{\parskip}{0pt}}
    \DefineVerbatimEnvironment{Highlighting}{Verbatim}{commandchars=\\\{\}}
    % Add ',fontsize=\small' for more characters per line
    \newenvironment{Shaded}{}{}
    \newcommand{\KeywordTok}[1]{\textcolor[rgb]{0.00,0.44,0.13}{\textbf{{#1}}}}
    \newcommand{\DataTypeTok}[1]{\textcolor[rgb]{0.56,0.13,0.00}{{#1}}}
    \newcommand{\DecValTok}[1]{\textcolor[rgb]{0.25,0.63,0.44}{{#1}}}
    \newcommand{\BaseNTok}[1]{\textcolor[rgb]{0.25,0.63,0.44}{{#1}}}
    \newcommand{\FloatTok}[1]{\textcolor[rgb]{0.25,0.63,0.44}{{#1}}}
    \newcommand{\CharTok}[1]{\textcolor[rgb]{0.25,0.44,0.63}{{#1}}}
    \newcommand{\StringTok}[1]{\textcolor[rgb]{0.25,0.44,0.63}{{#1}}}
    \newcommand{\CommentTok}[1]{\textcolor[rgb]{0.38,0.63,0.69}{\textit{{#1}}}}
    \newcommand{\OtherTok}[1]{\textcolor[rgb]{0.00,0.44,0.13}{{#1}}}
    \newcommand{\AlertTok}[1]{\textcolor[rgb]{1.00,0.00,0.00}{\textbf{{#1}}}}
    \newcommand{\FunctionTok}[1]{\textcolor[rgb]{0.02,0.16,0.49}{{#1}}}
    \newcommand{\RegionMarkerTok}[1]{{#1}}
    \newcommand{\ErrorTok}[1]{\textcolor[rgb]{1.00,0.00,0.00}{\textbf{{#1}}}}
    \newcommand{\NormalTok}[1]{{#1}}
    
    % Additional commands for more recent versions of Pandoc
    \newcommand{\ConstantTok}[1]{\textcolor[rgb]{0.53,0.00,0.00}{{#1}}}
    \newcommand{\SpecialCharTok}[1]{\textcolor[rgb]{0.25,0.44,0.63}{{#1}}}
    \newcommand{\VerbatimStringTok}[1]{\textcolor[rgb]{0.25,0.44,0.63}{{#1}}}
    \newcommand{\SpecialStringTok}[1]{\textcolor[rgb]{0.73,0.40,0.53}{{#1}}}
    \newcommand{\ImportTok}[1]{{#1}}
    \newcommand{\DocumentationTok}[1]{\textcolor[rgb]{0.73,0.13,0.13}{\textit{{#1}}}}
    \newcommand{\AnnotationTok}[1]{\textcolor[rgb]{0.38,0.63,0.69}{\textbf{\textit{{#1}}}}}
    \newcommand{\CommentVarTok}[1]{\textcolor[rgb]{0.38,0.63,0.69}{\textbf{\textit{{#1}}}}}
    \newcommand{\VariableTok}[1]{\textcolor[rgb]{0.10,0.09,0.49}{{#1}}}
    \newcommand{\ControlFlowTok}[1]{\textcolor[rgb]{0.00,0.44,0.13}{\textbf{{#1}}}}
    \newcommand{\OperatorTok}[1]{\textcolor[rgb]{0.40,0.40,0.40}{{#1}}}
    \newcommand{\BuiltInTok}[1]{{#1}}
    \newcommand{\ExtensionTok}[1]{{#1}}
    \newcommand{\PreprocessorTok}[1]{\textcolor[rgb]{0.74,0.48,0.00}{{#1}}}
    \newcommand{\AttributeTok}[1]{\textcolor[rgb]{0.49,0.56,0.16}{{#1}}}
    \newcommand{\InformationTok}[1]{\textcolor[rgb]{0.38,0.63,0.69}{\textbf{\textit{{#1}}}}}
    \newcommand{\WarningTok}[1]{\textcolor[rgb]{0.38,0.63,0.69}{\textbf{\textit{{#1}}}}}
    
    
    % Define a nice break command that doesn't care if a line doesn't already
    % exist.
    \def\br{\hspace*{\fill} \\* }
    % Math Jax compatability definitions
    \def\gt{>}
    \def\lt{<}
    % Document parameters
    \title{recession-analysis}
    
    
    

    % Pygments definitions
    
\makeatletter
\def\PY@reset{\let\PY@it=\relax \let\PY@bf=\relax%
    \let\PY@ul=\relax \let\PY@tc=\relax%
    \let\PY@bc=\relax \let\PY@ff=\relax}
\def\PY@tok#1{\csname PY@tok@#1\endcsname}
\def\PY@toks#1+{\ifx\relax#1\empty\else%
    \PY@tok{#1}\expandafter\PY@toks\fi}
\def\PY@do#1{\PY@bc{\PY@tc{\PY@ul{%
    \PY@it{\PY@bf{\PY@ff{#1}}}}}}}
\def\PY#1#2{\PY@reset\PY@toks#1+\relax+\PY@do{#2}}

\expandafter\def\csname PY@tok@w\endcsname{\def\PY@tc##1{\textcolor[rgb]{0.73,0.73,0.73}{##1}}}
\expandafter\def\csname PY@tok@c\endcsname{\let\PY@it=\textit\def\PY@tc##1{\textcolor[rgb]{0.25,0.50,0.50}{##1}}}
\expandafter\def\csname PY@tok@cp\endcsname{\def\PY@tc##1{\textcolor[rgb]{0.74,0.48,0.00}{##1}}}
\expandafter\def\csname PY@tok@k\endcsname{\let\PY@bf=\textbf\def\PY@tc##1{\textcolor[rgb]{0.00,0.50,0.00}{##1}}}
\expandafter\def\csname PY@tok@kp\endcsname{\def\PY@tc##1{\textcolor[rgb]{0.00,0.50,0.00}{##1}}}
\expandafter\def\csname PY@tok@kt\endcsname{\def\PY@tc##1{\textcolor[rgb]{0.69,0.00,0.25}{##1}}}
\expandafter\def\csname PY@tok@o\endcsname{\def\PY@tc##1{\textcolor[rgb]{0.40,0.40,0.40}{##1}}}
\expandafter\def\csname PY@tok@ow\endcsname{\let\PY@bf=\textbf\def\PY@tc##1{\textcolor[rgb]{0.67,0.13,1.00}{##1}}}
\expandafter\def\csname PY@tok@nb\endcsname{\def\PY@tc##1{\textcolor[rgb]{0.00,0.50,0.00}{##1}}}
\expandafter\def\csname PY@tok@nf\endcsname{\def\PY@tc##1{\textcolor[rgb]{0.00,0.00,1.00}{##1}}}
\expandafter\def\csname PY@tok@nc\endcsname{\let\PY@bf=\textbf\def\PY@tc##1{\textcolor[rgb]{0.00,0.00,1.00}{##1}}}
\expandafter\def\csname PY@tok@nn\endcsname{\let\PY@bf=\textbf\def\PY@tc##1{\textcolor[rgb]{0.00,0.00,1.00}{##1}}}
\expandafter\def\csname PY@tok@ne\endcsname{\let\PY@bf=\textbf\def\PY@tc##1{\textcolor[rgb]{0.82,0.25,0.23}{##1}}}
\expandafter\def\csname PY@tok@nv\endcsname{\def\PY@tc##1{\textcolor[rgb]{0.10,0.09,0.49}{##1}}}
\expandafter\def\csname PY@tok@no\endcsname{\def\PY@tc##1{\textcolor[rgb]{0.53,0.00,0.00}{##1}}}
\expandafter\def\csname PY@tok@nl\endcsname{\def\PY@tc##1{\textcolor[rgb]{0.63,0.63,0.00}{##1}}}
\expandafter\def\csname PY@tok@ni\endcsname{\let\PY@bf=\textbf\def\PY@tc##1{\textcolor[rgb]{0.60,0.60,0.60}{##1}}}
\expandafter\def\csname PY@tok@na\endcsname{\def\PY@tc##1{\textcolor[rgb]{0.49,0.56,0.16}{##1}}}
\expandafter\def\csname PY@tok@nt\endcsname{\let\PY@bf=\textbf\def\PY@tc##1{\textcolor[rgb]{0.00,0.50,0.00}{##1}}}
\expandafter\def\csname PY@tok@nd\endcsname{\def\PY@tc##1{\textcolor[rgb]{0.67,0.13,1.00}{##1}}}
\expandafter\def\csname PY@tok@s\endcsname{\def\PY@tc##1{\textcolor[rgb]{0.73,0.13,0.13}{##1}}}
\expandafter\def\csname PY@tok@sd\endcsname{\let\PY@it=\textit\def\PY@tc##1{\textcolor[rgb]{0.73,0.13,0.13}{##1}}}
\expandafter\def\csname PY@tok@si\endcsname{\let\PY@bf=\textbf\def\PY@tc##1{\textcolor[rgb]{0.73,0.40,0.53}{##1}}}
\expandafter\def\csname PY@tok@se\endcsname{\let\PY@bf=\textbf\def\PY@tc##1{\textcolor[rgb]{0.73,0.40,0.13}{##1}}}
\expandafter\def\csname PY@tok@sr\endcsname{\def\PY@tc##1{\textcolor[rgb]{0.73,0.40,0.53}{##1}}}
\expandafter\def\csname PY@tok@ss\endcsname{\def\PY@tc##1{\textcolor[rgb]{0.10,0.09,0.49}{##1}}}
\expandafter\def\csname PY@tok@sx\endcsname{\def\PY@tc##1{\textcolor[rgb]{0.00,0.50,0.00}{##1}}}
\expandafter\def\csname PY@tok@m\endcsname{\def\PY@tc##1{\textcolor[rgb]{0.40,0.40,0.40}{##1}}}
\expandafter\def\csname PY@tok@gh\endcsname{\let\PY@bf=\textbf\def\PY@tc##1{\textcolor[rgb]{0.00,0.00,0.50}{##1}}}
\expandafter\def\csname PY@tok@gu\endcsname{\let\PY@bf=\textbf\def\PY@tc##1{\textcolor[rgb]{0.50,0.00,0.50}{##1}}}
\expandafter\def\csname PY@tok@gd\endcsname{\def\PY@tc##1{\textcolor[rgb]{0.63,0.00,0.00}{##1}}}
\expandafter\def\csname PY@tok@gi\endcsname{\def\PY@tc##1{\textcolor[rgb]{0.00,0.63,0.00}{##1}}}
\expandafter\def\csname PY@tok@gr\endcsname{\def\PY@tc##1{\textcolor[rgb]{1.00,0.00,0.00}{##1}}}
\expandafter\def\csname PY@tok@ge\endcsname{\let\PY@it=\textit}
\expandafter\def\csname PY@tok@gs\endcsname{\let\PY@bf=\textbf}
\expandafter\def\csname PY@tok@gp\endcsname{\let\PY@bf=\textbf\def\PY@tc##1{\textcolor[rgb]{0.00,0.00,0.50}{##1}}}
\expandafter\def\csname PY@tok@go\endcsname{\def\PY@tc##1{\textcolor[rgb]{0.53,0.53,0.53}{##1}}}
\expandafter\def\csname PY@tok@gt\endcsname{\def\PY@tc##1{\textcolor[rgb]{0.00,0.27,0.87}{##1}}}
\expandafter\def\csname PY@tok@err\endcsname{\def\PY@bc##1{\setlength{\fboxsep}{0pt}\fcolorbox[rgb]{1.00,0.00,0.00}{1,1,1}{\strut ##1}}}
\expandafter\def\csname PY@tok@kc\endcsname{\let\PY@bf=\textbf\def\PY@tc##1{\textcolor[rgb]{0.00,0.50,0.00}{##1}}}
\expandafter\def\csname PY@tok@kd\endcsname{\let\PY@bf=\textbf\def\PY@tc##1{\textcolor[rgb]{0.00,0.50,0.00}{##1}}}
\expandafter\def\csname PY@tok@kn\endcsname{\let\PY@bf=\textbf\def\PY@tc##1{\textcolor[rgb]{0.00,0.50,0.00}{##1}}}
\expandafter\def\csname PY@tok@kr\endcsname{\let\PY@bf=\textbf\def\PY@tc##1{\textcolor[rgb]{0.00,0.50,0.00}{##1}}}
\expandafter\def\csname PY@tok@bp\endcsname{\def\PY@tc##1{\textcolor[rgb]{0.00,0.50,0.00}{##1}}}
\expandafter\def\csname PY@tok@fm\endcsname{\def\PY@tc##1{\textcolor[rgb]{0.00,0.00,1.00}{##1}}}
\expandafter\def\csname PY@tok@vc\endcsname{\def\PY@tc##1{\textcolor[rgb]{0.10,0.09,0.49}{##1}}}
\expandafter\def\csname PY@tok@vg\endcsname{\def\PY@tc##1{\textcolor[rgb]{0.10,0.09,0.49}{##1}}}
\expandafter\def\csname PY@tok@vi\endcsname{\def\PY@tc##1{\textcolor[rgb]{0.10,0.09,0.49}{##1}}}
\expandafter\def\csname PY@tok@vm\endcsname{\def\PY@tc##1{\textcolor[rgb]{0.10,0.09,0.49}{##1}}}
\expandafter\def\csname PY@tok@sa\endcsname{\def\PY@tc##1{\textcolor[rgb]{0.73,0.13,0.13}{##1}}}
\expandafter\def\csname PY@tok@sb\endcsname{\def\PY@tc##1{\textcolor[rgb]{0.73,0.13,0.13}{##1}}}
\expandafter\def\csname PY@tok@sc\endcsname{\def\PY@tc##1{\textcolor[rgb]{0.73,0.13,0.13}{##1}}}
\expandafter\def\csname PY@tok@dl\endcsname{\def\PY@tc##1{\textcolor[rgb]{0.73,0.13,0.13}{##1}}}
\expandafter\def\csname PY@tok@s2\endcsname{\def\PY@tc##1{\textcolor[rgb]{0.73,0.13,0.13}{##1}}}
\expandafter\def\csname PY@tok@sh\endcsname{\def\PY@tc##1{\textcolor[rgb]{0.73,0.13,0.13}{##1}}}
\expandafter\def\csname PY@tok@s1\endcsname{\def\PY@tc##1{\textcolor[rgb]{0.73,0.13,0.13}{##1}}}
\expandafter\def\csname PY@tok@mb\endcsname{\def\PY@tc##1{\textcolor[rgb]{0.40,0.40,0.40}{##1}}}
\expandafter\def\csname PY@tok@mf\endcsname{\def\PY@tc##1{\textcolor[rgb]{0.40,0.40,0.40}{##1}}}
\expandafter\def\csname PY@tok@mh\endcsname{\def\PY@tc##1{\textcolor[rgb]{0.40,0.40,0.40}{##1}}}
\expandafter\def\csname PY@tok@mi\endcsname{\def\PY@tc##1{\textcolor[rgb]{0.40,0.40,0.40}{##1}}}
\expandafter\def\csname PY@tok@il\endcsname{\def\PY@tc##1{\textcolor[rgb]{0.40,0.40,0.40}{##1}}}
\expandafter\def\csname PY@tok@mo\endcsname{\def\PY@tc##1{\textcolor[rgb]{0.40,0.40,0.40}{##1}}}
\expandafter\def\csname PY@tok@ch\endcsname{\let\PY@it=\textit\def\PY@tc##1{\textcolor[rgb]{0.25,0.50,0.50}{##1}}}
\expandafter\def\csname PY@tok@cm\endcsname{\let\PY@it=\textit\def\PY@tc##1{\textcolor[rgb]{0.25,0.50,0.50}{##1}}}
\expandafter\def\csname PY@tok@cpf\endcsname{\let\PY@it=\textit\def\PY@tc##1{\textcolor[rgb]{0.25,0.50,0.50}{##1}}}
\expandafter\def\csname PY@tok@c1\endcsname{\let\PY@it=\textit\def\PY@tc##1{\textcolor[rgb]{0.25,0.50,0.50}{##1}}}
\expandafter\def\csname PY@tok@cs\endcsname{\let\PY@it=\textit\def\PY@tc##1{\textcolor[rgb]{0.25,0.50,0.50}{##1}}}

\def\PYZbs{\char`\\}
\def\PYZus{\char`\_}
\def\PYZob{\char`\{}
\def\PYZcb{\char`\}}
\def\PYZca{\char`\^}
\def\PYZam{\char`\&}
\def\PYZlt{\char`\<}
\def\PYZgt{\char`\>}
\def\PYZsh{\char`\#}
\def\PYZpc{\char`\%}
\def\PYZdl{\char`\$}
\def\PYZhy{\char`\-}
\def\PYZsq{\char`\'}
\def\PYZdq{\char`\"}
\def\PYZti{\char`\~}
% for compatibility with earlier versions
\def\PYZat{@}
\def\PYZlb{[}
\def\PYZrb{]}
\makeatother


    % Exact colors from NB
    \definecolor{incolor}{rgb}{0.0, 0.0, 0.5}
    \definecolor{outcolor}{rgb}{0.545, 0.0, 0.0}



    
    % Prevent overflowing lines due to hard-to-break entities
    \sloppy 
    % Setup hyperref package
    \hypersetup{
      breaklinks=true,  % so long urls are correctly broken across lines
      colorlinks=true,
      urlcolor=urlcolor,
      linkcolor=linkcolor,
      citecolor=citecolor,
      }
    % Slightly bigger margins than the latex defaults
    
    \geometry{verbose,tmargin=1in,bmargin=1in,lmargin=1in,rmargin=1in}
    
    

    \begin{document}
    
    
    \maketitle
    
    

    
    \section{Table of Contents}\label{table-of-contents}

{1~~}Introduction

{2~~}Data

{3~~}Recession extraction

{3.1~~}Vogel \& Kroll

{3.2~~}Brutsaert

{3.3~~}Kirchner

{4~~}Recession analysis

{4.1~~}Single recession curves

{4.2~~}Master recession curves

{5~~}References

    \section{Introduction}\label{introduction}

In the absence of precipitation, periods of recession of a streamflow
are generally considered to be representative of the storage-discharge
relationship of a watershed. In particular, if the surface storage
(lakes, glaciers, ...) is negligible and under homogeneous
hydrogeological conditions, most of the flow comes from the underground
water resources and the analysis of the recession periods is supposed to
reflect the storage-discharge relationship of the aquifer. As a result,
the analysis of recession periods is an open door to the study of the
dynamic (and possibly physical properties) of the aquifer. Usually,
recession analysis are preformed on daily data.

However, recession analysis methods (RAM) have to deal with technical
challenges to both extract recession events and analyze them. In terms
of extraction, it is generally observed that after peak flows due to
direct runoff, the early recession remains under the influence of other
components before being exclusively driven by the aquifer discharge. For
this reason, RAMs often discard this early period with respect to some
heuristic rules. Regarding the investigation of the recovered recession
segments, the analysis is usually model-based in the sense that
assumptions about the dynamics of the recession are made. In most case,
it consists in considering the aquifer as linear or nonlinear reservoirs
with no other abstraction than discharge.

This notebook introduces three different RAMs based on the one compared
in Stoelzle et al., 2013. The later paper also questions whether the
recession characteristics are truly characteristics since different RAMs
may yield to different outcomes. As always in hydrology, the use of
simple and recognized but highly reductionist concepts (eg. return
period, time of concentration, ...) can be questioned. The choice is up
to the expert based on his knowledge of how, where, when and for what
purpose a tool works and on his ability of selecting the best tool or
combination of tools for a given task.

    \begin{Verbatim}[commandchars=\\\{\}]
{\color{incolor}In [{\color{incolor}31}]:} \PY{k+kn}{import} \PY{n+nn}{pandas} \PY{k}{as} \PY{n+nn}{pd}
         \PY{k+kn}{import} \PY{n+nn}{numpy} \PY{k}{as} \PY{n+nn}{np}
         \PY{k+kn}{import} \PY{n+nn}{matplotlib}\PY{n+nn}{.}\PY{n+nn}{pyplot} \PY{k}{as} \PY{n+nn}{plt}
         \PY{k+kn}{from} \PY{n+nn}{scipy}\PY{n+nn}{.}\PY{n+nn}{ndimage}\PY{n+nn}{.}\PY{n+nn}{measurements} \PY{k}{import} \PY{n}{label}
         \PY{o}{\PYZpc{}}\PY{k}{matplotlib} inline
\end{Verbatim}


    \section{Data}\label{data}

Considering a daily streamflow time series, \(Q = Q(t)\).

    \begin{Verbatim}[commandchars=\\\{\}]
{\color{incolor}In [{\color{incolor}150}]:} \PY{c+c1}{\PYZsh{} Loading and resampling to daily mean discharge}
          \PY{n}{q} \PY{o}{=} \PY{n}{pd}\PY{o}{.}\PY{n}{read\PYZus{}csv}\PY{p}{(}\PY{l+s+s1}{\PYZsq{}}\PY{l+s+s1}{data/discharge.csv}\PY{l+s+s1}{\PYZsq{}}\PY{p}{,} \PY{n}{parse\PYZus{}dates}\PY{o}{=}\PY{k+kc}{True}\PY{p}{,} \PY{n}{index\PYZus{}col}\PY{o}{=}\PY{l+m+mi}{0}\PY{p}{)}\PY{o}{.}\PY{n}{resample}\PY{p}{(}\PY{l+s+s1}{\PYZsq{}}\PY{l+s+s1}{D}\PY{l+s+s1}{\PYZsq{}}\PY{p}{)}\PY{o}{.}\PY{n}{mean}\PY{p}{(}\PY{p}{)}
          \PY{n}{q}\PY{o}{.}\PY{n}{plot}\PY{p}{(}\PY{n}{figsize}\PY{o}{=}\PY{p}{(}\PY{l+m+mi}{10}\PY{p}{,}\PY{l+m+mi}{4}\PY{p}{)}\PY{p}{)}
          \PY{n}{q}\PY{o}{.}\PY{n}{describe}\PY{p}{(}\PY{p}{)}
\end{Verbatim}


\begin{Verbatim}[commandchars=\\\{\}]
{\color{outcolor}Out[{\color{outcolor}150}]:}        Discharge [cms]
          count      7306.000000
          mean          4.288430
          std           4.649659
          min           0.150000
          25\%           1.237397
          50\%           2.812920
          75\%           5.607426
          max          62.117227
\end{Verbatim}
            
    \begin{center}
    \adjustimage{max size={0.9\linewidth}{0.9\paperheight}}{output_4_1.png}
    \end{center}
    { \hspace*{\fill} \\}
    
    \section{Recession extraction}\label{recession-extraction}

Note that in Stoezle et al. the original streamflow data is pre-whitened
with \(Q = (Q_t + Q_{t+1})/2\). This whitening is suggested in Brusaert
and Nieber. However, I do not apply it in this notebook.

\subsection{Vogel \& Kroll}\label{vogel-kroll}

Vogel and Kroll (1996) suggested to extract recession events from the
decreasing parts of 3-day moving averages of streamflow. These segments
must have a minimum length of 10 continuous days, and the decline in
discharge for two consecutive data values has to be smaller than 30 \%.
Furthermore the first 30 \% of every recession segment is excluded to
avoid the influence of storm and surface runoff at the beginning of
streamflow recessions.

Applying the rolling mean can be easily done with pandas python library.
We define recession event as negative slope. The percentage of decline
is computed. Recession events are then individually labeled using the
scipy label function. We then need to iterate through the labeled
recession events to extract the beginning, the end and the duration.

    \begin{Verbatim}[commandchars=\\\{\}]
{\color{incolor}In [{\color{incolor}98}]:} \PY{c+c1}{\PYZsh{} Applying 3 days moving average}
         \PY{n}{q}\PY{p}{[}\PY{l+s+s1}{\PYZsq{}}\PY{l+s+s1}{MA3}\PY{l+s+s1}{\PYZsq{}}\PY{p}{]} \PY{o}{=} \PY{n}{q}\PY{p}{[}\PY{l+s+s1}{\PYZsq{}}\PY{l+s+s1}{Discharge [cms]}\PY{l+s+s1}{\PYZsq{}}\PY{p}{]}\PY{o}{.}\PY{n}{rolling}\PY{p}{(}\PY{n}{window}\PY{o}{=}\PY{l+m+mi}{3}\PY{p}{)}\PY{o}{.}\PY{n}{mean}\PY{p}{(}\PY{p}{)}
         \PY{c+c1}{\PYZsh{} Define coarse recession event}
         \PY{n}{q}\PY{p}{[}\PY{l+s+s1}{\PYZsq{}}\PY{l+s+s1}{is\PYZus{}recession}\PY{l+s+s1}{\PYZsq{}}\PY{p}{]} \PY{o}{=} \PY{n}{np}\PY{o}{.}\PY{n}{logical\PYZus{}and}\PY{p}{(}\PY{n}{q}\PY{p}{[}\PY{l+s+s1}{\PYZsq{}}\PY{l+s+s1}{MA3}\PY{l+s+s1}{\PYZsq{}}\PY{p}{]}\PY{o}{.}\PY{n}{diff}\PY{p}{(}\PY{p}{)} \PY{o}{\PYZlt{}} \PY{l+m+mi}{0}\PY{p}{,} \PY{n}{q}\PY{p}{[}\PY{l+s+s1}{\PYZsq{}}\PY{l+s+s1}{Discharge [cms]}\PY{l+s+s1}{\PYZsq{}}\PY{p}{]}\PY{o}{.}\PY{n}{diff}\PY{p}{(}\PY{p}{)} \PY{o}{\PYZlt{}} \PY{l+m+mi}{0}\PY{p}{)}
         \PY{c+c1}{\PYZsh{} Compute the decline ratio}
         \PY{n}{q}\PY{p}{[}\PY{l+s+s1}{\PYZsq{}}\PY{l+s+s1}{decline\PYZus{}pct}\PY{l+s+s1}{\PYZsq{}}\PY{p}{]} \PY{o}{=} \PY{n+nb}{abs}\PY{p}{(}\PY{n}{q}\PY{p}{[}\PY{l+s+s1}{\PYZsq{}}\PY{l+s+s1}{Discharge [cms]}\PY{l+s+s1}{\PYZsq{}}\PY{p}{]}\PY{o}{.}\PY{n}{diff}\PY{p}{(}\PY{p}{)}\PY{p}{)} \PY{o}{/} \PY{n}{q}\PY{p}{[}\PY{l+s+s1}{\PYZsq{}}\PY{l+s+s1}{Discharge [cms]}\PY{l+s+s1}{\PYZsq{}}\PY{p}{]}
         \PY{c+c1}{\PYZsh{} Label each recession event using scipy label}
         \PY{n}{evt}\PY{p}{,} \PY{n}{nlab} \PY{o}{=} \PY{n}{label}\PY{p}{(}\PY{n}{q}\PY{p}{[}\PY{l+s+s1}{\PYZsq{}}\PY{l+s+s1}{is\PYZus{}recession}\PY{l+s+s1}{\PYZsq{}}\PY{p}{]}\PY{p}{)}
         \PY{n}{q}\PY{p}{[}\PY{l+s+s1}{\PYZsq{}}\PY{l+s+s1}{evt}\PY{l+s+s1}{\PYZsq{}}\PY{p}{]} \PY{o}{=} \PY{n}{evt}
         \PY{c+c1}{\PYZsh{} Creation of a event dataframe}
         \PY{n}{evtdf1} \PY{o}{=} \PY{n}{pd}\PY{o}{.}\PY{n}{DataFrame}\PY{p}{(}\PY{n}{columns}\PY{o}{=}\PY{p}{[}\PY{l+s+s1}{\PYZsq{}}\PY{l+s+s1}{start}\PY{l+s+s1}{\PYZsq{}}\PY{p}{,} \PY{l+s+s1}{\PYZsq{}}\PY{l+s+s1}{rstart}\PY{l+s+s1}{\PYZsq{}}\PY{p}{,} \PY{l+s+s1}{\PYZsq{}}\PY{l+s+s1}{end}\PY{l+s+s1}{\PYZsq{}}\PY{p}{,} \PY{l+s+s1}{\PYZsq{}}\PY{l+s+s1}{duration}\PY{l+s+s1}{\PYZsq{}}\PY{p}{,} \PY{l+s+s1}{\PYZsq{}}\PY{l+s+s1}{valid}\PY{l+s+s1}{\PYZsq{}}\PY{p}{]}\PY{p}{)}
         \PY{k}{for} \PY{n}{e} \PY{o+ow}{in} \PY{n+nb}{range}\PY{p}{(}\PY{l+m+mi}{1}\PY{p}{,}\PY{n}{nlab}\PY{o}{+}\PY{l+m+mi}{1}\PY{p}{)}\PY{p}{:}
             \PY{n}{sel} \PY{o}{=} \PY{n}{q}\PY{p}{[}\PY{n}{q}\PY{p}{[}\PY{l+s+s1}{\PYZsq{}}\PY{l+s+s1}{evt}\PY{l+s+s1}{\PYZsq{}}\PY{p}{]} \PY{o}{==} \PY{n}{e}\PY{p}{]}
             \PY{n}{evtdf1}\PY{o}{.}\PY{n}{loc}\PY{p}{[}\PY{n}{e}\PY{p}{,} \PY{l+s+s1}{\PYZsq{}}\PY{l+s+s1}{start}\PY{l+s+s1}{\PYZsq{}}\PY{p}{]} \PY{o}{=} \PY{n}{sel}\PY{o}{.}\PY{n}{index}\PY{p}{[}\PY{l+m+mi}{0}\PY{p}{]} 
             \PY{n}{evtdf1}\PY{o}{.}\PY{n}{loc}\PY{p}{[}\PY{n}{e}\PY{p}{,} \PY{l+s+s1}{\PYZsq{}}\PY{l+s+s1}{end}\PY{l+s+s1}{\PYZsq{}}\PY{p}{]} \PY{o}{=} \PY{n}{sel}\PY{o}{.}\PY{n}{index}\PY{p}{[}\PY{o}{\PYZhy{}}\PY{l+m+mi}{1}\PY{p}{]}
             \PY{n}{duration} \PY{o}{=} \PY{n}{sel}\PY{o}{.}\PY{n}{index}\PY{p}{[}\PY{o}{\PYZhy{}}\PY{l+m+mi}{1}\PY{p}{]} \PY{o}{\PYZhy{}} \PY{n}{sel}\PY{o}{.}\PY{n}{index}\PY{p}{[}\PY{l+m+mi}{0}\PY{p}{]}
             \PY{n}{evtdf1}\PY{o}{.}\PY{n}{loc}\PY{p}{[}\PY{n}{e}\PY{p}{,} \PY{l+s+s1}{\PYZsq{}}\PY{l+s+s1}{duration}\PY{l+s+s1}{\PYZsq{}}\PY{p}{]} \PY{o}{=} \PY{n}{duration}
             \PY{n}{evtdf1}\PY{o}{.}\PY{n}{loc}\PY{p}{[}\PY{n}{e}\PY{p}{,} \PY{l+s+s1}{\PYZsq{}}\PY{l+s+s1}{rstart}\PY{l+s+s1}{\PYZsq{}}\PY{p}{]} \PY{o}{=} \PY{n}{sel}\PY{o}{.}\PY{n}{index}\PY{p}{[}\PY{l+m+mi}{0}\PY{p}{]} \PY{o}{+} \PY{n}{duration}\PY{o}{.}\PY{n}{days} \PY{o}{/}\PY{o}{/} \PY{l+m+mi}{3}
             \PY{n}{evtdf1}\PY{o}{.}\PY{n}{loc}\PY{p}{[}\PY{n}{e}\PY{p}{,} \PY{l+s+s1}{\PYZsq{}}\PY{l+s+s1}{valid}\PY{l+s+s1}{\PYZsq{}}\PY{p}{]} \PY{o}{=} \PY{n+nb}{all}\PY{p}{(}\PY{n}{sel}\PY{p}{[}\PY{l+s+s1}{\PYZsq{}}\PY{l+s+s1}{decline\PYZus{}pct}\PY{l+s+s1}{\PYZsq{}}\PY{p}{]} \PY{o}{\PYZlt{}} \PY{l+m+mf}{0.3}\PY{p}{)}
         \PY{n}{evtdf1} \PY{o}{=} \PY{n}{evtdf1}\PY{p}{[}\PY{n}{evtdf1}\PY{p}{[}\PY{l+s+s1}{\PYZsq{}}\PY{l+s+s1}{valid}\PY{l+s+s1}{\PYZsq{}}\PY{p}{]}\PY{p}{]}
         \PY{n}{evtdf1} \PY{o}{=} \PY{n}{evtdf1}\PY{p}{[}\PY{n}{evtdf1}\PY{p}{[}\PY{l+s+s1}{\PYZsq{}}\PY{l+s+s1}{duration}\PY{l+s+s1}{\PYZsq{}}\PY{p}{]}\PY{o}{\PYZgt{}}\PY{o}{=} \PY{n}{pd}\PY{o}{.}\PY{n}{to\PYZus{}timedelta}\PY{p}{(}\PY{l+s+s1}{\PYZsq{}}\PY{l+s+s1}{10 Days}\PY{l+s+s1}{\PYZsq{}}\PY{p}{)}\PY{p}{]}
         \PY{n}{evtdf1}\PY{o}{.}\PY{n}{reset\PYZus{}index}\PY{p}{(}\PY{n}{inplace}\PY{o}{=}\PY{k+kc}{True}\PY{p}{,} \PY{n}{drop}\PY{o}{=}\PY{k+kc}{True}\PY{p}{)}
         \PY{n+nb}{print}\PY{p}{(}\PY{l+s+s1}{\PYZsq{}}\PY{l+s+s1}{Number of recession events: }\PY{l+s+si}{\PYZob{}\PYZcb{}}\PY{l+s+s1}{\PYZsq{}}\PY{o}{.}\PY{n}{format}\PY{p}{(}\PY{n+nb}{len}\PY{p}{(}\PY{n}{evtdf1}\PY{p}{)}\PY{p}{)}\PY{p}{)}
         \PY{n}{evtdf1}\PY{o}{.}\PY{n}{head}\PY{p}{(}\PY{p}{)}
\end{Verbatim}


    \begin{Verbatim}[commandchars=\\\{\}]
Number of recession events: 64

    \end{Verbatim}

\begin{Verbatim}[commandchars=\\\{\}]
{\color{outcolor}Out[{\color{outcolor}98}]:}                  start               rstart                  end  \textbackslash{}
         0  1969-02-28 00:00:00  1969-03-03 00:00:00  1969-03-10 00:00:00   
         1  1969-03-16 00:00:00  1969-03-20 00:00:00  1969-03-30 00:00:00   
         2  1969-12-28 00:00:00  1969-12-31 00:00:00  1970-01-08 00:00:00   
         3  1970-12-06 00:00:00  1970-12-10 00:00:00  1970-12-18 00:00:00   
         4  1971-02-21 00:00:00  1971-02-25 00:00:00  1971-03-07 00:00:00   
         
                    duration valid  
         0  10 days 00:00:00  True  
         1  14 days 00:00:00  True  
         2  11 days 00:00:00  True  
         3  12 days 00:00:00  True  
         4  14 days 00:00:00  True  
\end{Verbatim}
            
    \begin{Verbatim}[commandchars=\\\{\}]
{\color{incolor}In [{\color{incolor}184}]:} \PY{n}{fig} \PY{o}{=} \PY{n}{plt}\PY{o}{.}\PY{n}{figure}\PY{p}{(}\PY{p}{)}
          \PY{n}{ax} \PY{o}{=} \PY{n}{fig}\PY{o}{.}\PY{n}{gca}\PY{p}{(}\PY{p}{)}
          \PY{k}{for} \PY{n}{i}\PY{p}{,} \PY{n}{e} \PY{o+ow}{in} \PY{n}{evtdf1}\PY{o}{.}\PY{n}{iterrows}\PY{p}{(}\PY{p}{)}\PY{p}{:}
              \PY{n}{sel} \PY{o}{=} \PY{n}{q}\PY{o}{.}\PY{n}{loc}\PY{p}{[}\PY{n}{e}\PY{p}{[}\PY{l+s+s1}{\PYZsq{}}\PY{l+s+s1}{rstart}\PY{l+s+s1}{\PYZsq{}}\PY{p}{]}\PY{p}{:}\PY{n}{e}\PY{p}{[}\PY{l+s+s1}{\PYZsq{}}\PY{l+s+s1}{end}\PY{l+s+s1}{\PYZsq{}}\PY{p}{]}\PY{p}{]}
              \PY{n}{ax}\PY{o}{.}\PY{n}{semilogy}\PY{p}{(}\PY{n}{sel}\PY{p}{[}\PY{l+s+s1}{\PYZsq{}}\PY{l+s+s1}{Discharge [cms]}\PY{l+s+s1}{\PYZsq{}}\PY{p}{]}\PY{o}{.}\PY{n}{values}\PY{p}{,} \PY{n}{color}\PY{o}{=}\PY{l+s+s1}{\PYZsq{}}\PY{l+s+s1}{k}\PY{l+s+s1}{\PYZsq{}}\PY{p}{,} \PY{n}{lw}\PY{o}{=}\PY{l+m+mf}{1.}\PY{p}{,} \PY{n}{alpha}\PY{o}{=}\PY{l+m+mf}{0.5}\PY{p}{)}
          \PY{n}{ax}\PY{o}{.}\PY{n}{set\PYZus{}ylabel}\PY{p}{(}\PY{l+s+s1}{\PYZsq{}}\PY{l+s+s1}{\PYZdl{}log(dQ/dt)\PYZdl{}}\PY{l+s+s1}{\PYZsq{}}\PY{p}{,} \PY{n}{fontsize}\PY{o}{=}\PY{l+m+mi}{15}\PY{p}{)}
          \PY{n}{ax}\PY{o}{.}\PY{n}{set\PYZus{}xlabel}\PY{p}{(}\PY{l+s+s1}{\PYZsq{}}\PY{l+s+s1}{Recession time (Days)}\PY{l+s+s1}{\PYZsq{}}\PY{p}{,} \PY{n}{fontsize}\PY{o}{=}\PY{l+m+mi}{15}\PY{p}{)}
          \PY{k}{for} \PY{n}{side} \PY{o+ow}{in} \PY{p}{[}\PY{l+s+s1}{\PYZsq{}}\PY{l+s+s1}{right}\PY{l+s+s1}{\PYZsq{}}\PY{p}{,}\PY{l+s+s1}{\PYZsq{}}\PY{l+s+s1}{top}\PY{l+s+s1}{\PYZsq{}}\PY{p}{]}\PY{p}{:}
              \PY{n}{ax}\PY{o}{.}\PY{n}{spines}\PY{p}{[}\PY{n}{side}\PY{p}{]}\PY{o}{.}\PY{n}{set\PYZus{}visible}\PY{p}{(}\PY{k+kc}{False}\PY{p}{)}
\end{Verbatim}


    \begin{center}
    \adjustimage{max size={0.9\linewidth}{0.9\paperheight}}{output_7_0.png}
    \end{center}
    { \hspace*{\fill} \\}
    
    \subsection{Brutsaert}\label{brutsaert}

The Brutsaert method (Brutsaert and Nieber, 1977) extracts recession
parts from hydrograph with the following criteria: no values with
positive or zero dQ/dt are allowed and three data points after the last
and two data points before the first positive or zero dQ/dt are
eliminated. Additionally a fourth data point is excluded after major
events. Due to no further specification, Stoelzle \emph{et al} defined
in their study a major event as streamflow values greater than the 30 \%
exceedance frequency (Q30) during the period of record.Further on, the
Brutsaert method eliminates data points followed by values with a larger
−dQ/dt in order to exclude sudden anomalies and the ups and downs of
dQ/dt values during a recession.

This methods requires to evaluate the Q30 value. To do so, we will rely
on the percentile 70 of the data and on the \emph{flow duration curve}
(FDC) that pictures the percentage of time a streamflow discharged is
equaled or exceeded.

    \begin{Verbatim}[commandchars=\\\{\}]
{\color{incolor}In [{\color{incolor}188}]:} \PY{n}{q} \PY{o}{=} \PY{n}{pd}\PY{o}{.}\PY{n}{read\PYZus{}csv}\PY{p}{(}\PY{l+s+s1}{\PYZsq{}}\PY{l+s+s1}{data/discharge.csv}\PY{l+s+s1}{\PYZsq{}}\PY{p}{,} \PY{n}{parse\PYZus{}dates}\PY{o}{=}\PY{k+kc}{True}\PY{p}{,} \PY{n}{index\PYZus{}col}\PY{o}{=}\PY{l+m+mi}{0}\PY{p}{)}\PY{o}{.}\PY{n}{resample}\PY{p}{(}\PY{l+s+s1}{\PYZsq{}}\PY{l+s+s1}{D}\PY{l+s+s1}{\PYZsq{}}\PY{p}{)}\PY{o}{.}\PY{n}{mean}\PY{p}{(}\PY{p}{)}
          \PY{n}{qt} \PY{o}{=} \PY{n}{q}\PY{p}{[}\PY{l+s+s1}{\PYZsq{}}\PY{l+s+s1}{Discharge [cms]}\PY{l+s+s1}{\PYZsq{}}\PY{p}{]}
          \PY{n}{partition} \PY{o}{=} \PY{n}{np}\PY{o}{.}\PY{n}{linspace}\PY{p}{(}\PY{n+nb}{max}\PY{p}{(}\PY{n}{qt}\PY{p}{)}\PY{p}{,} \PY{n+nb}{min}\PY{p}{(}\PY{n}{qt}\PY{p}{)}\PY{p}{,} \PY{l+m+mi}{1000}\PY{p}{)}
          \PY{n}{pctge} \PY{o}{=} \PY{n}{np}\PY{o}{.}\PY{n}{zeros}\PY{p}{(}\PY{n+nb}{len}\PY{p}{(}\PY{n}{partition}\PY{p}{)}\PY{p}{)}
          
          \PY{k}{for} \PY{n}{i}\PY{p}{,} \PY{n}{qp} \PY{o+ow}{in} \PY{n+nb}{enumerate}\PY{p}{(}\PY{n}{partition}\PY{p}{)}\PY{p}{:}
              \PY{n}{pctge}\PY{p}{[}\PY{n}{i}\PY{p}{]} \PY{o}{=} \PY{n}{np}\PY{o}{.}\PY{n}{sum}\PY{p}{(}\PY{n}{np}\PY{o}{.}\PY{n}{greater\PYZus{}equal}\PY{p}{(}\PY{n}{qt}\PY{p}{,}\PY{n}{qp}\PY{p}{)}\PY{p}{)}\PY{o}{/}\PY{n+nb}{len}\PY{p}{(}\PY{n}{qt}\PY{p}{)}\PY{o}{*}\PY{l+m+mi}{100}
              
          \PY{n}{q30} \PY{o}{=} \PY{n}{np}\PY{o}{.}\PY{n}{percentile}\PY{p}{(}\PY{n}{qt}\PY{p}{,} \PY{l+m+mi}{70}\PY{p}{)}
          
          \PY{n}{fig} \PY{o}{=} \PY{n}{plt}\PY{o}{.}\PY{n}{figure}\PY{p}{(}\PY{p}{)}
          \PY{n}{ax} \PY{o}{=} \PY{n}{fig}\PY{o}{.}\PY{n}{gca}\PY{p}{(}\PY{p}{)}
          \PY{n}{ax}\PY{o}{.}\PY{n}{set\PYZus{}title}\PY{p}{(}\PY{l+s+s1}{\PYZsq{}}\PY{l+s+s1}{Flow duration curve}\PY{l+s+s1}{\PYZsq{}}\PY{p}{,} \PY{n}{fontsize}\PY{o}{=}\PY{l+m+mi}{15}\PY{p}{)}
          \PY{n}{ax}\PY{o}{.}\PY{n}{semilogy}\PY{p}{(}\PY{n}{pctge}\PY{p}{,} \PY{n}{partition}\PY{p}{,} \PY{l+s+s1}{\PYZsq{}}\PY{l+s+s1}{\PYZhy{}\PYZhy{}k}\PY{l+s+s1}{\PYZsq{}}\PY{p}{,} \PY{n}{lw}\PY{o}{=}\PY{l+m+mi}{1}\PY{p}{)}
          \PY{n}{ax}\PY{o}{.}\PY{n}{scatter}\PY{p}{(}\PY{l+m+mi}{30}\PY{p}{,} \PY{n}{q30}\PY{p}{,} \PY{n}{marker}\PY{o}{=}\PY{l+s+s1}{\PYZsq{}}\PY{l+s+s1}{s}\PY{l+s+s1}{\PYZsq{}}\PY{p}{,} \PY{n}{color}\PY{o}{=}\PY{l+s+s1}{\PYZsq{}}\PY{l+s+s1}{k}\PY{l+s+s1}{\PYZsq{}}\PY{p}{)}
          \PY{n}{ax}\PY{o}{.}\PY{n}{text}\PY{p}{(}\PY{l+m+mi}{33}\PY{p}{,} \PY{n}{q30} \PY{o}{+} \PY{l+m+mi}{1}\PY{p}{,} \PY{l+s+s1}{\PYZsq{}}\PY{l+s+s1}{\PYZdl{}Q = }\PY{l+s+si}{\PYZob{}:.2f\PYZcb{}}\PY{l+s+s1}{\PYZdl{} cms}\PY{l+s+s1}{\PYZsq{}}\PY{o}{.}\PY{n}{format}\PY{p}{(}\PY{n}{q30}\PY{p}{)}\PY{p}{)}
          \PY{n}{ax}\PY{o}{.}\PY{n}{vlines}\PY{p}{(}\PY{l+m+mi}{30}\PY{p}{,} \PY{l+m+mi}{0}\PY{p}{,} \PY{n}{q30}\PY{p}{,} \PY{n}{lw}\PY{o}{=}\PY{l+m+mf}{0.5}\PY{p}{)}
          \PY{n}{ax}\PY{o}{.}\PY{n}{hlines}\PY{p}{(}\PY{n}{q30}\PY{p}{,} \PY{n}{xmin} \PY{o}{=} \PY{l+m+mi}{0}\PY{p}{,} \PY{n}{xmax}\PY{o}{=} \PY{l+m+mi}{30}\PY{p}{,} \PY{n}{lw}\PY{o}{=}\PY{l+m+mf}{0.5}\PY{p}{)}
          \PY{n}{ax}\PY{o}{.}\PY{n}{set\PYZus{}xlabel}\PY{p}{(}\PY{l+s+s1}{\PYZsq{}}\PY{l+s+s1}{Time equaled or exceeded (}\PY{l+s+s1}{\PYZpc{}}\PY{l+s+s1}{)}\PY{l+s+s1}{\PYZsq{}}\PY{p}{,} \PY{n}{fontsize}\PY{o}{=}\PY{l+m+mi}{15}\PY{p}{)}
          \PY{n}{ax}\PY{o}{.}\PY{n}{set\PYZus{}ylabel}\PY{p}{(}\PY{l+s+s1}{\PYZsq{}}\PY{l+s+s1}{Discharge [cms]}\PY{l+s+s1}{\PYZsq{}}\PY{p}{,} \PY{n}{fontsize}\PY{o}{=}\PY{l+m+mi}{15}\PY{p}{)}
          \PY{c+c1}{\PYZsh{}ax.grid(which=\PYZsq{}minor\PYZsq{}, lw=0.5, ls=\PYZsq{}:\PYZsq{})}
          \PY{k}{for} \PY{n}{side} \PY{o+ow}{in} \PY{p}{[}\PY{l+s+s1}{\PYZsq{}}\PY{l+s+s1}{right}\PY{l+s+s1}{\PYZsq{}}\PY{p}{,}\PY{l+s+s1}{\PYZsq{}}\PY{l+s+s1}{top}\PY{l+s+s1}{\PYZsq{}}\PY{p}{]}\PY{p}{:}
              \PY{n}{ax}\PY{o}{.}\PY{n}{spines}\PY{p}{[}\PY{n}{side}\PY{p}{]}\PY{o}{.}\PY{n}{set\PYZus{}visible}\PY{p}{(}\PY{k+kc}{False}\PY{p}{)}
\end{Verbatim}


    \begin{center}
    \adjustimage{max size={0.9\linewidth}{0.9\paperheight}}{output_9_0.png}
    \end{center}
    { \hspace*{\fill} \\}
    
    \begin{Verbatim}[commandchars=\\\{\}]
{\color{incolor}In [{\color{incolor}231}]:} \PY{c+c1}{\PYZsh{} Loading and resampling to daily mean discharge}
          \PY{n}{q} \PY{o}{=} \PY{n}{pd}\PY{o}{.}\PY{n}{read\PYZus{}csv}\PY{p}{(}\PY{l+s+s1}{\PYZsq{}}\PY{l+s+s1}{data/discharge.csv}\PY{l+s+s1}{\PYZsq{}}\PY{p}{,} \PY{n}{parse\PYZus{}dates}\PY{o}{=}\PY{k+kc}{True}\PY{p}{,} \PY{n}{index\PYZus{}col}\PY{o}{=}\PY{l+m+mi}{0}\PY{p}{)}\PY{o}{.}\PY{n}{resample}\PY{p}{(}\PY{l+s+s1}{\PYZsq{}}\PY{l+s+s1}{D}\PY{l+s+s1}{\PYZsq{}}\PY{p}{)}\PY{o}{.}\PY{n}{mean}\PY{p}{(}\PY{p}{)}
          \PY{n}{q}\PY{p}{[}\PY{l+s+s1}{\PYZsq{}}\PY{l+s+s1}{Discharge [cms]}\PY{l+s+s1}{\PYZsq{}}\PY{p}{]} \PY{o}{=} \PY{p}{(}\PY{n}{q}\PY{p}{[}\PY{l+s+s1}{\PYZsq{}}\PY{l+s+s1}{Discharge [cms]}\PY{l+s+s1}{\PYZsq{}}\PY{p}{]} \PY{o}{+} \PY{n}{q}\PY{p}{[}\PY{l+s+s1}{\PYZsq{}}\PY{l+s+s1}{Discharge [cms]}\PY{l+s+s1}{\PYZsq{}}\PY{p}{]}\PY{o}{.}\PY{n}{shift}\PY{p}{(}\PY{p}{)}\PY{p}{)}\PY{o}{/}\PY{l+m+mi}{2}
          \PY{n}{q} \PY{o}{=} \PY{n}{q}\PY{o}{.}\PY{n}{iloc}\PY{p}{[}\PY{l+m+mi}{1}\PY{p}{:}\PY{p}{]}
          \PY{n}{q}\PY{p}{[}\PY{l+s+s1}{\PYZsq{}}\PY{l+s+s1}{is\PYZus{}recession}\PY{l+s+s1}{\PYZsq{}}\PY{p}{]} \PY{o}{=} \PY{n}{q}\PY{p}{[}\PY{l+s+s1}{\PYZsq{}}\PY{l+s+s1}{Discharge [cms]}\PY{l+s+s1}{\PYZsq{}}\PY{p}{]}\PY{o}{.}\PY{n}{diff}\PY{p}{(}\PY{p}{)} \PY{o}{\PYZlt{}} \PY{l+m+mi}{0}
          \PY{c+c1}{\PYZsh{} Computing Q30}
          \PY{n}{q30} \PY{o}{=} \PY{n}{np}\PY{o}{.}\PY{n}{percentile}\PY{p}{(}\PY{n}{q}\PY{p}{[}\PY{l+s+s1}{\PYZsq{}}\PY{l+s+s1}{Discharge [cms]}\PY{l+s+s1}{\PYZsq{}}\PY{p}{]}\PY{p}{,} \PY{l+m+mi}{70}\PY{p}{)}
          \PY{c+c1}{\PYZsh{} Label each recession event using scipy label}
          \PY{n}{evt}\PY{p}{,} \PY{n}{nlab} \PY{o}{=} \PY{n}{label}\PY{p}{(}\PY{n}{q}\PY{p}{[}\PY{l+s+s1}{\PYZsq{}}\PY{l+s+s1}{is\PYZus{}recession}\PY{l+s+s1}{\PYZsq{}}\PY{p}{]}\PY{p}{)}
          \PY{n}{q}\PY{p}{[}\PY{l+s+s1}{\PYZsq{}}\PY{l+s+s1}{evt}\PY{l+s+s1}{\PYZsq{}}\PY{p}{]} \PY{o}{=} \PY{n}{evt}
          \PY{n}{evtdf2} \PY{o}{=} \PY{n}{pd}\PY{o}{.}\PY{n}{DataFrame}\PY{p}{(}\PY{n}{columns}\PY{o}{=}\PY{p}{[}\PY{l+s+s1}{\PYZsq{}}\PY{l+s+s1}{start}\PY{l+s+s1}{\PYZsq{}}\PY{p}{,} \PY{l+s+s1}{\PYZsq{}}\PY{l+s+s1}{rstart}\PY{l+s+s1}{\PYZsq{}}\PY{p}{,} \PY{l+s+s1}{\PYZsq{}}\PY{l+s+s1}{end}\PY{l+s+s1}{\PYZsq{}}\PY{p}{,}\PY{l+s+s1}{\PYZsq{}}\PY{l+s+s1}{rend}\PY{l+s+s1}{\PYZsq{}}\PY{p}{,} \PY{l+s+s1}{\PYZsq{}}\PY{l+s+s1}{duration}\PY{l+s+s1}{\PYZsq{}}\PY{p}{,} \PY{l+s+s1}{\PYZsq{}}\PY{l+s+s1}{valid}\PY{l+s+s1}{\PYZsq{}}\PY{p}{]}\PY{p}{)}
          \PY{n}{evtdf2}\PY{p}{[}\PY{l+s+s1}{\PYZsq{}}\PY{l+s+s1}{valid}\PY{l+s+s1}{\PYZsq{}}\PY{p}{]} \PY{o}{=} \PY{k+kc}{False}
          \PY{k}{for} \PY{n}{e} \PY{o+ow}{in} \PY{n+nb}{range}\PY{p}{(}\PY{l+m+mi}{1}\PY{p}{,}\PY{n}{nlab}\PY{o}{+}\PY{l+m+mi}{1}\PY{p}{)}\PY{p}{:}
              \PY{n}{sel} \PY{o}{=} \PY{n}{q}\PY{p}{[}\PY{n}{q}\PY{p}{[}\PY{l+s+s1}{\PYZsq{}}\PY{l+s+s1}{evt}\PY{l+s+s1}{\PYZsq{}}\PY{p}{]} \PY{o}{==} \PY{n}{e}\PY{p}{]}
              \PY{n}{sel2} \PY{o}{=} \PY{n}{q}\PY{p}{[}\PY{n}{q}\PY{p}{[}\PY{l+s+s1}{\PYZsq{}}\PY{l+s+s1}{evt}\PY{l+s+s1}{\PYZsq{}}\PY{p}{]} \PY{o}{==} \PY{n}{e} \PY{o}{\PYZhy{}}\PY{l+m+mi}{1}\PY{p}{]}
              \PY{n}{stormevt} \PY{o}{=} \PY{n}{q}\PY{o}{.}\PY{n}{loc}\PY{p}{[}\PY{n}{sel2}\PY{o}{.}\PY{n}{index}\PY{p}{[}\PY{o}{\PYZhy{}}\PY{l+m+mi}{1}\PY{p}{]}\PY{p}{:}\PY{n}{sel}\PY{o}{.}\PY{n}{index}\PY{p}{[}\PY{l+m+mi}{0}\PY{p}{]}\PY{p}{]}
              \PY{n}{is\PYZus{}major} \PY{o}{=} \PY{n+nb}{any}\PY{p}{(}\PY{n}{stormevt}\PY{o}{\PYZgt{}}\PY{n}{q30}\PY{p}{)}
              \PY{n}{evtdf2}\PY{o}{.}\PY{n}{loc}\PY{p}{[}\PY{n}{e}\PY{p}{,} \PY{l+s+s1}{\PYZsq{}}\PY{l+s+s1}{start}\PY{l+s+s1}{\PYZsq{}}\PY{p}{]} \PY{o}{=} \PY{n}{sel}\PY{o}{.}\PY{n}{index}\PY{p}{[}\PY{l+m+mi}{0}\PY{p}{]} 
              \PY{n}{evtdf2}\PY{o}{.}\PY{n}{loc}\PY{p}{[}\PY{n}{e}\PY{p}{,} \PY{l+s+s1}{\PYZsq{}}\PY{l+s+s1}{end}\PY{l+s+s1}{\PYZsq{}}\PY{p}{]} \PY{o}{=} \PY{n}{sel}\PY{o}{.}\PY{n}{index}\PY{p}{[}\PY{o}{\PYZhy{}}\PY{l+m+mi}{1}\PY{p}{]}
              \PY{n}{duration} \PY{o}{=} \PY{n}{sel}\PY{o}{.}\PY{n}{index}\PY{p}{[}\PY{o}{\PYZhy{}}\PY{l+m+mi}{1}\PY{p}{]} \PY{o}{\PYZhy{}} \PY{n}{sel}\PY{o}{.}\PY{n}{index}\PY{p}{[}\PY{l+m+mi}{0}\PY{p}{]}
              \PY{n}{evtdf2}\PY{o}{.}\PY{n}{loc}\PY{p}{[}\PY{n}{e}\PY{p}{,} \PY{l+s+s1}{\PYZsq{}}\PY{l+s+s1}{duration}\PY{l+s+s1}{\PYZsq{}}\PY{p}{]} \PY{o}{=} \PY{n}{duration}
              \PY{n}{evtdf2}\PY{o}{.}\PY{n}{loc}\PY{p}{[}\PY{n}{e}\PY{p}{,} \PY{l+s+s1}{\PYZsq{}}\PY{l+s+s1}{rstart}\PY{l+s+s1}{\PYZsq{}}\PY{p}{]} \PY{o}{=} \PY{n}{sel}\PY{o}{.}\PY{n}{index}\PY{p}{[}\PY{l+m+mi}{0}\PY{p}{]} \PY{o}{+} \PY{l+m+mi}{3} \PY{o}{+} \PY{n+nb}{int}\PY{p}{(}\PY{n}{is\PYZus{}major}\PY{p}{)}
              \PY{n}{evtdf2}\PY{o}{.}\PY{n}{loc}\PY{p}{[}\PY{n}{e}\PY{p}{,} \PY{l+s+s1}{\PYZsq{}}\PY{l+s+s1}{rend}\PY{l+s+s1}{\PYZsq{}}\PY{p}{]} \PY{o}{=} \PY{n}{sel}\PY{o}{.}\PY{n}{index}\PY{p}{[}\PY{o}{\PYZhy{}}\PY{l+m+mi}{1}\PY{p}{]} \PY{o}{\PYZhy{}}\PY{l+m+mi}{2}
          \PY{n}{evtdf2}\PY{p}{[}\PY{l+s+s1}{\PYZsq{}}\PY{l+s+s1}{valid}\PY{l+s+s1}{\PYZsq{}}\PY{p}{]} \PY{o}{=} \PY{n}{evtdf2}\PY{p}{[}\PY{l+s+s1}{\PYZsq{}}\PY{l+s+s1}{rend}\PY{l+s+s1}{\PYZsq{}}\PY{p}{]}  \PY{o}{\PYZgt{}} \PY{n}{evtdf2}\PY{p}{[}\PY{l+s+s1}{\PYZsq{}}\PY{l+s+s1}{rstart}\PY{l+s+s1}{\PYZsq{}}\PY{p}{]}
          \PY{n}{evtdf2} \PY{o}{=} \PY{n}{evtdf2}\PY{p}{[}\PY{n}{evtdf2}\PY{p}{[}\PY{l+s+s1}{\PYZsq{}}\PY{l+s+s1}{valid}\PY{l+s+s1}{\PYZsq{}}\PY{p}{]}\PY{p}{]}
          \PY{n}{evtdf2}\PY{o}{.}\PY{n}{reset\PYZus{}index}\PY{p}{(}\PY{n}{inplace}\PY{o}{=}\PY{k+kc}{True}\PY{p}{,} \PY{n}{drop}\PY{o}{=}\PY{k+kc}{True}\PY{p}{)}
          \PY{n+nb}{print}\PY{p}{(}\PY{l+s+s1}{\PYZsq{}}\PY{l+s+s1}{Number of recession events: }\PY{l+s+si}{\PYZob{}\PYZcb{}}\PY{l+s+s1}{\PYZsq{}}\PY{o}{.}\PY{n}{format}\PY{p}{(}\PY{n+nb}{len}\PY{p}{(}\PY{n}{evtdf2}\PY{p}{)}\PY{p}{)}\PY{p}{)}
          \PY{n}{evtdf2}\PY{o}{.}\PY{n}{head}\PY{p}{(}\PY{p}{)}
\end{Verbatim}


    \begin{Verbatim}[commandchars=\\\{\}]
Number of recession events: 203

    \end{Verbatim}

\begin{Verbatim}[commandchars=\\\{\}]
{\color{outcolor}Out[{\color{outcolor}231}]:}                  start               rstart                  end  \textbackslash{}
          0  1969-01-05 00:00:00  1969-01-09 00:00:00  1969-01-12 00:00:00   
          1  1969-01-20 00:00:00  1969-01-24 00:00:00  1969-01-28 00:00:00   
          2  1969-02-04 00:00:00  1969-02-08 00:00:00  1969-02-11 00:00:00   
          3  1969-02-27 00:00:00  1969-03-03 00:00:00  1969-03-11 00:00:00   
          4  1969-03-15 00:00:00  1969-03-19 00:00:00  1969-03-30 00:00:00   
          
                            rend          duration  valid  
          0  1969-01-10 00:00:00   7 days 00:00:00   True  
          1  1969-01-26 00:00:00   8 days 00:00:00   True  
          2  1969-02-09 00:00:00   7 days 00:00:00   True  
          3  1969-03-09 00:00:00  12 days 00:00:00   True  
          4  1969-03-28 00:00:00  15 days 00:00:00   True  
\end{Verbatim}
            
    \begin{Verbatim}[commandchars=\\\{\}]
{\color{incolor}In [{\color{incolor}232}]:} \PY{n}{fig} \PY{o}{=} \PY{n}{plt}\PY{o}{.}\PY{n}{figure}\PY{p}{(}\PY{p}{)}
          \PY{n}{ax} \PY{o}{=} \PY{n}{fig}\PY{o}{.}\PY{n}{gca}\PY{p}{(}\PY{p}{)}
          \PY{k}{for} \PY{n}{i}\PY{p}{,} \PY{n}{e} \PY{o+ow}{in} \PY{n}{evtdf2}\PY{o}{.}\PY{n}{iterrows}\PY{p}{(}\PY{p}{)}\PY{p}{:}
              \PY{n}{sel} \PY{o}{=} \PY{n}{q}\PY{o}{.}\PY{n}{loc}\PY{p}{[}\PY{n}{e}\PY{p}{[}\PY{l+s+s1}{\PYZsq{}}\PY{l+s+s1}{rstart}\PY{l+s+s1}{\PYZsq{}}\PY{p}{]}\PY{p}{:}\PY{n}{e}\PY{p}{[}\PY{l+s+s1}{\PYZsq{}}\PY{l+s+s1}{end}\PY{l+s+s1}{\PYZsq{}}\PY{p}{]}\PY{p}{]}
              \PY{n}{ax}\PY{o}{.}\PY{n}{semilogy}\PY{p}{(}\PY{n}{sel}\PY{p}{[}\PY{l+s+s1}{\PYZsq{}}\PY{l+s+s1}{Discharge [cms]}\PY{l+s+s1}{\PYZsq{}}\PY{p}{]}\PY{o}{.}\PY{n}{values}\PY{p}{,} \PY{n}{color}\PY{o}{=}\PY{l+s+s1}{\PYZsq{}}\PY{l+s+s1}{k}\PY{l+s+s1}{\PYZsq{}}\PY{p}{,} \PY{n}{lw}\PY{o}{=}\PY{l+m+mf}{1.}\PY{p}{,} \PY{n}{alpha}\PY{o}{=}\PY{l+m+mf}{0.5}\PY{p}{)}
          \PY{n}{ax}\PY{o}{.}\PY{n}{set\PYZus{}ylabel}\PY{p}{(}\PY{l+s+s1}{\PYZsq{}}\PY{l+s+s1}{\PYZdl{}log(dQ/dt)\PYZdl{}}\PY{l+s+s1}{\PYZsq{}}\PY{p}{,} \PY{n}{fontsize}\PY{o}{=}\PY{l+m+mi}{15}\PY{p}{)}
          \PY{n}{ax}\PY{o}{.}\PY{n}{set\PYZus{}xlabel}\PY{p}{(}\PY{l+s+s1}{\PYZsq{}}\PY{l+s+s1}{Recession time (Days)}\PY{l+s+s1}{\PYZsq{}}\PY{p}{,} \PY{n}{fontsize}\PY{o}{=}\PY{l+m+mi}{15}\PY{p}{)}
          \PY{k}{for} \PY{n}{side} \PY{o+ow}{in} \PY{p}{[}\PY{l+s+s1}{\PYZsq{}}\PY{l+s+s1}{right}\PY{l+s+s1}{\PYZsq{}}\PY{p}{,}\PY{l+s+s1}{\PYZsq{}}\PY{l+s+s1}{top}\PY{l+s+s1}{\PYZsq{}}\PY{p}{]}\PY{p}{:}
              \PY{n}{ax}\PY{o}{.}\PY{n}{spines}\PY{p}{[}\PY{n}{side}\PY{p}{]}\PY{o}{.}\PY{n}{set\PYZus{}visible}\PY{p}{(}\PY{k+kc}{False}\PY{p}{)}
\end{Verbatim}


    \begin{center}
    \adjustimage{max size={0.9\linewidth}{0.9\paperheight}}{output_11_0.png}
    \end{center}
    { \hspace*{\fill} \\}
    
    Remember that later on, the Brusaert method do not allows higher -dQ/dt
values.

    \subsection{Kirchner}\label{kirchner}

    \section{Recession analysis}\label{recession-analysis}

\subsection{Single recession curves}\label{single-recession-curves}

\subsection{Master recession curves}\label{master-recession-curves}

    \section{References}\label{references}

\begin{itemize}
\item
  Brutsaert, W. and Nieber, J. L.: Regionalized drought flow hydrographs
  from a mature glaciated plateau, Water Resources Research, 13(3),
  637--643, doi:10.1029/WR013i003p00637, 1977.
\item
  Stoelzle, M., Stahl, K. and Weiler, M.: Are streamflow recession
  characteristics really characteristic?, Hydrol. Earth Syst. Sci.,
  17(2), 817--828, doi:10.5194/hess-17-817-2013, 2013.
\item
  Vogel, R. M. and Kroll, C. N.: Regional geohydrologic-geomorphic
  relationships for the estimation of low-flow statistics, Water
  Resources Research, 28(9), 2451--2458, doi:10.1029/92WR01007, 1992.
\end{itemize}


    % Add a bibliography block to the postdoc
    
    
    
    \end{document}
